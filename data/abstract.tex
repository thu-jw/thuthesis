% !TeX root = ../main.tex

% 中英文摘要和关键字

\begin{abstract}
  深度学习的方法在分类、检测等视觉任务中取得了很好的表现。但是,这些任务主要体现了模型的感知能力,而没有涉及到更高层次的推理能力。近年来,研究人员们越来越关注对模型推理能力的探索,并提出了一系列与视觉推理相关的任务,例如视觉问答(VQA)、视觉常识推理(VCA)等。然而在这些任务中,模型常常可以从问题中的自然语言直接或间接得到推理过程,而没有真正地学会如何进行推理。为了解决这个问题,本文基于教程类视频提出了一个全新的任务:视频溯因常识推理。在这个任务中,模型需要基于初始和结束状态的图片来选出之间的最可能发生的动作序列。在该任务中,模型的输入不包含任何自然语言,所以不会从问题本身获得推理过程的提示。
  
  本文首先在一个教程类视频数据集 COIN 的基础上进行二次标注,并构建了视频溯因常识推理数据集(VideoABC)。在该数据上,很多在视频理解方面最先进的模型都没有取得很好的效果。为此,本文提出了一个全新多级双重推理模块。该模块可以在不同的分辨率等级上依次进行步骤内和步骤间的推理。基于该模块构建的网络(HDR Net)在本文提出的数据集上取得了 86.1\% 的准确率,比基础模型高出 8\%. 

  本文的创新点主要有:
  \begin{itemize}
    \item 提出了一个全新的视觉推理任务;
    \item 构建了一个视觉溯因常识推理数据集 VideoABC;
    \item 提出了一个新的推理模块,并在 VideoABC 上取得了目前最高的准确率。
  \end{itemize}

  % 关键词用“英文逗号”分隔
  \thusetup{
    keywords = {溯因推理, 视觉推理, 计算机视觉, 深度学习},
  }
\end{abstract}

\begin{abstract*}
  An abstract of a dissertation is a summary and extraction of research work
  and contributions. Included in an abstract should be description of research
  topic and research objective, brief introduction to methodology and research
  process, and summarization of conclusion and contributions of the
  research. An abstract should be characterized by independence and clarity and
  carry identical information with the dissertation. It should be such that the
  general idea and major contributions of the dissertation are conveyed without
  reading the dissertation.

  An abstract should be concise and to the point. It is a misunderstanding to
  make an abstract an outline of the dissertation and words ``the first
  chapter'', ``the second chapter'' and the like should be avoided in the
  abstract.

  Key words are terms used in a dissertation for indexing, reflecting core
  information of the dissertation. An abstract may contain a maximum of 5 key
  words, with semi-colons used in between to separate one another.
  \thusetup{
    keywords* = {Abductive reasoning, Visual reasoning, Computer vision, Deep learning},
  }
\end{abstract*}
